%%%%%%%%%%%%%%%%%%%%%%%%%%%%%%%%%%%%%%%%%%%%%%%%%%%%%%%
% MatPlotLib and Random Cheat Sheet
%
% Edited by Michelle Cristina de Sousa Baltazar
%
% http://matplotlib.org/api/pyplot_summary.html
% http://matplotlib.org/users/pyplot_tutorial.html
%
%%%%%%%%%%%%%%%%%%%%%%%%%%%%%%%%%%%%%%%%%%%%%%%%%%%%%%%

\documentclass{article}
\usepackage[a4paper]{geometry}
\usepackage{url}
\usepackage{multicol}
\usepackage{amsmath}
\usepackage{amsfonts}
\usepackage{tikz}
\usetikzlibrary{decorations,arrows.meta}
\usetikzlibrary{decorations.pathreplacing}
\usetikzlibrary{decorations.pathmorphing}
\usepackage{amsmath,amssymb}
\usepackage{shortvrb}

\usepackage{colortbl}
\usepackage{xcolor}
\usepackage{mathtools}
\usepackage{amsmath,amssymb}
\usepackage{enumitem}

\title{Cheatsheet - \LaTeX}
\usepackage[brazilian]{babel}
\usepackage[utf8]{inputenc}
\usepackage{booktabs}

\advance\topmargin-.8in
\advance\textheight3in
\advance\textwidth3in
\advance\oddsidemargin-1in
\advance\evensidemargin-1.5in
\parindent0pt
\parskip2pt
\newcommand{\hr}{\centerline{\rule{3.5in}{1pt}}}
% \colorbox[HTML]{e4e4e4}{\makebox[\textwidth-2\fboxsep][l]{texto}
\begin{document}

\begin{center}{\huge{\textbf{Cheatsheet - \LaTeX~~~~~~~~~~~~}}}\\
  {\large ACM UPM~~~~~~~~http://t.me/acmlatex~~~~~~~~~~~~~~~~~~~~~~~}
\end{center}


\tikzstyle{mybox} = [draw=black, fill=white, very thick,
rectangle, rounded corners, inner sep=10pt, inner ysep=10pt]
\tikzstyle{fancytitle} =[fill=black, text=white, font=\bfseries]
% ------------ CONTEÚDO CAIXA MatPlotLib ---------------
\begin{tikzpicture}
  \node [mybox] (box){%
    \begin{minipage}{0.85\textwidth}

      \begin{tabular}{p{0.5\textwidth} p{0.4\textwidth}}
        \verb|% Comentario en LateX                          | &                  \\
        \verb|% Caracteres reservados: # $ % ^ & _ { } ~ \   | &                  \\
        \verb|% \comando[opción1, opción2]{arg1}{arg2}       | &                  \\
        \verb|texto normal                                   | & texto normal     \\
        \verb|comportamiento     de~~~~~los    espacios      | & comportamiento     de~~~~~los    espacios                 \\
        \verb|\LaTeX            % Si empieza por \, comando  | & \LaTeX           \\
        \verb|\sigma            % Símbolos matemáticos       | & $\sigma$         \\
        \verb|{\large Formato}  % Contextos entre llaves     | & {\large Formato} \\
        \verb|$a = 2\pi r$      % Contexto matemático        | & $a = 2\pi r$     \\
        \verb|\`{a}             % Algún caracter especial    | & \`{a}            \\
        \verb|\u{a}             % Algún caracter especial    | & \u{a}            \\
      \end{tabular}
      % Comentario en LaTeX
    \end{minipage}
  };
  % ------------ CAIXA PRELIMINARES ---------------------
  \node[fancytitle, right=10pt] at (box.north west) {Sintaxis básica};
\end{tikzpicture}


% ------------ CONTEÚDO CAIXA RANDOM ---------------
\begin{tikzpicture}
  \node [mybox] (box){%
    \begin{minipage}[!]{0.85\textwidth}
      \begin{multicols*}{2}
        \begin{minipage}{0.2\textwidth}
\begin{verbatim}
\begin{equation}
  x = \frac{b^2 \pm\sqrt{b^2 - 4ac}}{2a}
\end{equation}
\end{verbatim}
        \end{minipage}

        \begin{minipage}[!]{.45\linewidth}
\begin{verbatim}


\[y = \sqrt{(x_2 - x_1)^2 + (y_2 - y_1)^2}\]
\end{verbatim}
        \end{minipage}

        \begin{minipage}[!]{.45\linewidth}
          \begin{equation}
            x = \frac{b^2 \pm\sqrt{b^2 - 4ac}}{2a}
          \end{equation}
        \end{minipage}

        \begin{minipage}[!]{.45\linewidth}
          \[y = \sqrt{(x_2 - x_1)^2 + (y_2 - y_1)^2}\]
        \end{minipage}
      \end{multicols*}
    \end{minipage}
  };
  % ------------ CAIXA RANDOM ---------------------
  \node[fancytitle, right=10pt] at (box.north west) {Contextos matemáticos};
\end{tikzpicture}

% ------------ CONTEUDO EXEMPLO BASICO ---------------------
\begin{tikzpicture}
  \node [mybox] (box){%
    \begin{minipage}{0.85\textwidth}
      \begin{multicols*}{2}
        \begin{minipage}[!]{.45\linewidth}
\begin{verbatim}


\begin{itemize}
\item item1
\item item2
\item item3
\end{itemize}
\end{verbatim}
        \end{minipage}

        \begin{minipage}[!]{.45\linewidth}
\begin{verbatim}


\begin{enumerate}
\item Item 1
\item Item 2
\item Item 3
\end{enumerate}
\end{verbatim}
        \end{minipage}

        \begin{minipage}[!]{.45\linewidth}
\begin{verbatim}


\begin{description}
\item[Etiqueta 1] Descripción
\item[Etiqueta 2] Descripción que ocupa
  más de una línea, por lo que salta.
\item[Etiqueta 3] Descripción
\end{description}
\end{verbatim}

        \end{minipage}

        \begin{minipage}[!]{.45\linewidth}
          ~\\
          \begin{itemize}
          \item item1
          \item item2
          \item item3
          \end{itemize}
        \end{minipage}

        \begin{minipage}[!]{.45\linewidth}
          ~\\
          \begin{enumerate}
          \item Item 1
          \item Item 2
          \item Item 3
          \end{enumerate}
        \end{minipage}

        \begin{minipage}[!]{.9\linewidth}
          ~\\
          \begin{description}
          \item[Etiqueta 1] Descripción
          \item[Etiqueta 2] Descripción que ocupa más de una línea,
            por lo que salta.
          \item[Etiqueta 3] Descripción
          \end{description}
        \end{minipage}
      \end{multicols*}
    \end{minipage}
  };
  % ------------ EXEMPLO BASICO BOX ---------------------
  \node[fancytitle, right=10pt] at (box.north west) {Contextos estructurados};
\end{tikzpicture}

\begin{tikzpicture}
  \node [mybox] (box){%
    \begin{minipage}{0.85\textwidth}
      \begin{tabular}{p{0.5\textwidth} p{0.4\textwidth}}
        \verb|Fuente normal                  | & Fuente normal \\
        \verb|\emph{Énfasis, cursiva}        | & \emph{Énfasis, cursiva} \\
        \verb|\texttt{Monospace, teletype}   | & \texttt{Monospace, teletype} \\
        \verb|\textsf{Sans-serif}|             & \textsl{Sans-serif} \\
        \verb|\textsc{Small Captital}|         & \textsc{Small Captital} \\
        \verb|\uppercase{Uppercase} |          & \uppercase{Uppercase} \\
      \end{tabular}
    \end{minipage}
  };
  % ------------ NAME ---------------------
  \node[fancytitle, right=10pt] at (box.north west) {Formato de fuentes};
\end{tikzpicture}


\begin{tikzpicture}
  \node [mybox] (box){%
    \begin{minipage}{0.85\textwidth}
      \begin{multicols*}{2}
        \begin{minipage}[!]{.65\linewidth}
\begin{verbatim}


\includegraphics[width=.4\textwidth]{../img/acmblack.png}


\includegraphics[width=.4\textwidth]{../img/qrcode.png}
\end{verbatim}
        \end{minipage}

        \begin{minipage}[!]{.35\linewidth}
          \begin{center}
            \includegraphics[width=.4\textwidth]{../img/acmblack.png}
            \hspace{15pt}
            \includegraphics[width=.4\textwidth]{../img/qrcode.png}
          \end{center}
        \end{minipage}
      \end{multicols*}
    \end{minipage}
  };
  \node[fancytitle, right=10pt] at (box.north west) {Imágenes};
\end{tikzpicture}

\begin{tikzpicture}
  \node [mybox] (box){%
    \begin{minipage}{0.85\textwidth}
      \begin{multicols*}{2}
        \begin{minipage}[!]{.45\linewidth}
\begin{verbatim}

\begin{tabular}{|l |l |l |}  % Otras letras: c (center) r(right) p(paragraph)
  \hline
  Nemónico & Operandos & Ciclos \\ \hline
  MOVE.B   & (An), Dn  & 8(2/0) \\
  ADD.B    & Dn, dadr  & 4(1/0)+\\
  \hline
\end{tabular}
\end{verbatim}
        \end{minipage}

        \begin{minipage}[!]{.45\linewidth}
\begin{verbatim}


% \usepackage{booktabs}
\begin{tabular}{l l l}
  \toprule
  Nemónico & Operandos & Ciclos \\ \midrule
  MOVE.B   & (An), Dn  & 8(2/0) \\
  ADD.B    & Dn, dadr  & 4(1/0)+\\
  \bottomrule
\end{tabular}
\end{verbatim}
        \end{minipage}


        \begin{minipage}[!]{.45\linewidth}
          ~\\
          ~\\
          ~\\
          ~\\
          \begin{tabular}{|l |l | l |}
            \hline
            Nemónico & Operandos & Ciclos \\ \hline
            MOVE.B   & (An), Dn  & 8(2/0) \\
            ADD.B    & Dn, dadr  & 4(1/0)+\\
            \hline
          \end{tabular}
        \end{minipage}

        \begin{minipage}[!]{.45\linewidth}
          ~\\
          ~\\
          ~\\
          ~\\
          ~\\
          \begin{tabular}{l l l}
            \toprule
            Nemónico & Operandos & Ciclos \\ \midrule
            MOVE.B   & (An), Dn  & 8(2/0) \\
            ADD.B    & Dn, dadr  & 4(1/0)+\\
            \bottomrule
          \end{tabular}
        \end{minipage}
      \end{multicols*}
    \end{minipage}
  };
  \node[fancytitle, right=10pt] at (box.north west) {Tablas};
\end{tikzpicture}

\begin{tikzpicture}
  \node [mybox] (box){%
    \begin{minipage}{0.85\textwidth}
      \begin{multicols*}{2}
        \begin{minipage}[!]{.45\linewidth}
\begin{verbatim}

\section{Sección 1}

\subsection{Subsección 1.1}
\subsubsection{Subsubsección 1.1.1}

\subsection{Subsección 1.2}

\section{Sección 2}
\end{verbatim}
        \end{minipage}

        \begin{minipage}[!]{.8\linewidth}
          \section{Sección 1}

          \subsection{Subsección 1.1}
          \subsubsection{Subsubsección 1.1.1}

          \subsection{Subsección 1.2}

          \section{Sección 2}
        \end{minipage}
      \end{multicols*}

    \end{minipage}

  };
  \node[fancytitle, right=10pt] at (box.north west) {Secciones};
\end{tikzpicture}


\begin{tikzpicture}
  \node [mybox] (box){%
    \begin{minipage}{0.85\textwidth}
\begin{verbatim}

\documentclass[11pt]{article} %% letter, report, chapter

\usepackage[spanish]{babel}
\usepackage[utf8]{inputenc}
\usepackage[a4paper]{geometry}
\usepackage[hidelinks=true]{hyperref}

\usepackage{graphicx}

\title{Título del documento}
\author{ACM UPM}
\date{\today}


\begin{document}

\maketitle

%% Aquí se empieza a escribir el contenido

\end{document}

\end{verbatim}
    \end{minipage}

  };
  \node[fancytitle, right=10pt] at (box.north west) {Documento};
\end{tikzpicture}

\begin{tikzpicture}
  \node [mybox] (box){%
    \begin{minipage}{0.85\textwidth}
\begin{verbatim}












\end{verbatim}
    \end{minipage}
  };
  \node[fancytitle, right=10pt] at (box.north west) {Notas};
\end{tikzpicture}
\end{document}