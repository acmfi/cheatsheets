%%%%%%%%%%%%%%%%%%%%%%%%%%%%%%%%%%%%%%%%%%%%%%%%%%%%%%%
% Cheatsheet template - ACM UPM
%
% Original by Michelle Cristina de Sousa Baltazar
%
%%%%%%%%%%%%%%%%%%%%%%%%%%%%%%%%%%%%%%%%%%%%%%%%%%%%%%%

\documentclass{article}
\usepackage[landscape]{geometry}
\usepackage{url}
\usepackage{multicol}
\usepackage{amsmath}
\usepackage{amsfonts}
\usepackage{tikz}
\usepackage{amsmath,amssymb}
\usepackage{colortbl}
\usepackage{xcolor}
\usepackage{mathtools}
\usepackage{amsmath,amssymb}
\usepackage{enumitem}
\usepackage[spanish]{babel}
\usepackage[utf8]{inputenc}
\usepackage{titling}


% Parámetros sin tocar de la plantilla original
\advance\topmargin-.8in
\advance\textheight3in
\advance\textwidth3in
\advance\oddsidemargin-1.5in
\advance\evensidemargin-1.5in
\parindent0pt
\parskip2pt

\tikzstyle{mybox} = [draw=black, fill=white, very thick,
rectangle, rounded corners, inner sep=10pt, inner ysep=10pt]
\tikzstyle{fancytitle} =[fill=black, text=white, font=\bfseries]


\title{Cheatsheet - Template}
\author{ACM UPM}

\begin{document}

\begin{center}
  {\huge{\textbf{\thetitle}}}\\ % del paquete titling
  {\large \theauthor}           % del paquete titling
\end{center}

\begin{multicols*}{3}           % Divide la hoja en 3 columnas

  \begin{tikzpicture}           % Crea una caja
    \node [mybox] (box){%
      % Al dividirse en 3, tomamos algo menos de un tercio de la página
      \begin{minipage}{0.3\textwidth}
        Contenido de la caja.

        Al ir rellenando las cajas se distribuirán en las 3 columnas creadas.
      \end{minipage}
    };
    % Título de la caja
    \node[fancytitle, right=10pt] at (box.north west) {Título de la caja};
  \end{tikzpicture}

  \begin{tikzpicture}
    \node [mybox] (box){%
      \begin{minipage}{0.3\textwidth} % Contenido
      \end{minipage}
    };
    \node[fancytitle, right=10pt] at (box.north west) {};  % Título
  \end{tikzpicture}

\end{multicols*}
\end{document}