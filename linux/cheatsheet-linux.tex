%%%%%%%%%%%%%%%%%%%%%%%%%%%%%%%%%%%%%%%%%%%%%%%%%%%%%%%
% MatPlotLib and Random Cheat Sheet
%
% Edited by Michelle Cristina de Sousa Baltazar
%
% http://matplotlib.org/api/pyplot_summary.html
% http://matplotlib.org/users/pyplot_tutorial.html
%
%%%%%%%%%%%%%%%%%%%%%%%%%%%%%%%%%%%%%%%%%%%%%%%%%%%%%%%

\documentclass{article}
\usepackage[landscape]{geometry}
\usepackage{url}
\usepackage{multicol}
\usepackage{amsmath}
\usepackage{amsfonts}
\usepackage{tikz}
\usetikzlibrary{decorations,arrows.meta}
\usetikzlibrary{decorations.pathreplacing}
\usetikzlibrary{decorations.pathmorphing}
\usepackage{amsmath,amssymb}

\usepackage{colortbl}
\usepackage{xcolor}
\usepackage{mathtools}
\usepackage{amsmath,amssymb}
\usepackage{enumitem}

\title{Cheatsheet - Linux}
\usepackage[brazilian]{babel}
\usepackage[utf8]{inputenc}

\advance\topmargin-.8in
\advance\textheight3in
\advance\textwidth3in
\advance\oddsidemargin-1.5in
\advance\evensidemargin-1.5in
\parindent0pt
\parskip2pt
\newcommand{\hr}{\centerline{\rule{3.5in}{1pt}}}
%\colorbox[HTML]{e4e4e4}{\makebox[\textwidth-2\fboxsep][l]{texto}
\begin{document}

\begin{center}{\huge{\textbf{Cheatsheet - Linux}}}\\
{\large ACM UPM}
\end{center}
\begin{multicols*}{3}

\tikzstyle{mybox} = [draw=black, fill=white, very thick,
    rectangle, rounded corners, inner sep=10pt, inner ysep=10pt]
    \tikzstyle{fancytitle} =[fill=black, text=white, font=\bfseries]
%------------ CONTEÚDO CAIXA MatPlotLib ---------------
\begin{tikzpicture}
\node [mybox] (box){%
\begin{minipage}{0.3\textwidth}
\begin{verbatim}
.           # Directorio actual
..          # Directorio superior
/           # Directorio raíz
~           # HOME del usuario (virgulilla)
\end{verbatim}
\end{minipage}
};
%------------ CAIXA PRELIMINARES ---------------------
\node[fancytitle, right=10pt] at (box.north west) {Definiciones};
\end{tikzpicture}


%------------ CONTEÚDO CAIXA RANDOM ---------------
\begin{tikzpicture}
\node [mybox] (box){%
    \begin{minipage}{0.3\textwidth}
\begin{verbatim}
pwd         # Print Working Directory

cd carpeta  # Change Directory (a carpeta)
cd ..       # Moverse al directorio superior
cd          # Moverse a HOME
cd ~        # Moverse a HOME
cd /        # Ir a la raíz del sistema
cd z/y/x    # Cambio a x, siguiendo la ruta

ls          # Listar directorio actual (ls .)
ls ..       # Listar directorio ..
ls carpeta  # Lista directorio "carpeta"
ls -a       # Listar también archivos ocultos
ls -l       # Listar con formato
ls -lah     # Todos los archivos formateados

mv f1 f2    # Mover el fichero f1 a f2
            # (f1 desaparece)
            # (Si f2 existe se sobreescribe)

cp f1 f2    # Copiar el fichero f1 en f2
            # (f1 sigue existiendo)
            # (Si f2 existe se sobreescribe)

mv -r d1 d2 # Mueve recursivamente d1 a d2
cp -r d1 d2 # Copia recursivamente d1 en d2

rm f1       # Elimina el fichero f1
rmdir d1    # Elimina el directorio vacío d1

rm -r d1    # Elimina el directorio d1 y
            #  todo su contenido

################ DANGER ZONE ################
rm -rf --no-preserve-root /
            # NO HAGAS ESTO SOBRE TODO EN UEFI
            #  PUEDES ROMPER TODA TU MÁQUINA
#############################################
\end{verbatim}
\end{minipage}
};
%------------ CAIXA RANDOM ---------------------
\node[fancytitle, right=10pt] at (box.north west) {Movimiento básico};
\end{tikzpicture}

%------------ CONTEUDO EXEMPLO BASICO ---------------------
\begin{tikzpicture}
\node [mybox] (box){%
    \begin{minipage}{0.3\textwidth}
      \begin{tabular}{l c c c l}
          &          4 &          2 &          1 & \\
        1 &            &            & \texttt{x} & Ejecución \\
        2 &            & \texttt{w} &            & Escritura \\
        3 &            & \texttt{w} & \texttt{x} & Escritura + Ejecución \\
        4 & \texttt{r} &            &            & Lectura   \\
        5 & \texttt{r} &            & \texttt{x} & Lectura + Ejecución \\
        6 & \texttt{r} & \texttt{w} &  \texttt{} & Lectura + Escritura \\
        7 & \texttt{r} & \texttt{w} & \texttt{x} & Lectura + Escritura
                                                   + Ejecución \\
      \end{tabular}

      \begin{tikzpicture}
        \node [] (rt) at (0,0) {\texttt{-} $\underbrace{\underbrace{\texttt{r -
              x}}_{\text{usuario}}~\underbrace{\texttt{r w
              -}}_{\text{grupo}}~\underbrace{\texttt{r - -}}_{\text{otros}}}_{\text{Permisos}}$};
        \node [] (tipo-fichero)   at (-3.6,0) {\footnotesize
          \begin{tabular}{r l}
            \multicolumn{2}{l}{Tipo de fichero} \\
            \texttt{-} & Fichero regular \\
            \texttt{d} & Directorio \\
          \end{tabular}
        };

        \node [] (rwx) at (1.3,1.3) {\footnotesize
          \begin{tabular}{l}
            \texttt{lectura}\\
            \texttt{ejecución}\\
            \texttt{escritura}
          \end{tabular}
        };

        \draw [->] ([xshift=-.5cm]tipo-fichero.east) -| ([xshift=-1.6cm,yshift=1.3cm]rt.south);

        \draw [->] ([xshift=.5cm,yshift=.35cm]rwx.west)  -| ([xshift=-1.32cm,yshift=-.25cm]rt.north);
        \draw [->] ([xshift=.5cm,yshift=.0cm]rwx.west)   -| ([xshift=-.58cm,yshift=-.25cm]rt.north);
        \draw [->] ([xshift=.5cm,yshift=-.35cm]rwx.west) -| ([xshift=.15cm,yshift=-.25cm]rt.north);
      \end{tikzpicture}

\begin{verbatim}
chmod 755 file  # Permisos rwx/r-x/r-x
chmod  +x file  # Añade permiso de ejecución
chmod o-x file  # Quita ejecución a "otros"
\end{verbatim}

    \end{minipage}
};
%------------ EXEMPLO BASICO BOX ---------------------
\node[fancytitle, right=10pt] at (box.north west) {Permisos de ficheros};
\end{tikzpicture}

\begin{tikzpicture}
\node [mybox] (box){%
    \begin{minipage}{0.3\textwidth}
\begin{verbatim}
man ls    # Entrada del manual para ls
man man   # Entrada del manual para man

man 1     # Ejecutable o mandato de shell
man 2     # Llamadas al sistemas
man 3     # Llamadas a bibliotecas
man 4     # Ficheros especiales
man 5     # Formatos y convenciones
man 6     # Juegos
man 7     # Diversas
man 8     # Administración de sistemas
man 9     # Rutinas del núcleo
\end{verbatim}
    \end{minipage}
};
%------------ NAME ---------------------
\node[fancytitle, right=10pt] at (box.north west) {Manuales};
\end{tikzpicture}

\begin{tikzpicture}

  \node[] at (0,2) {}; % Such padding

  \node[] at (0,0) {
    \begin{minipage}{0.3\textwidth}
      \begin{center}
        \includegraphics[width=0.3\textwidth]{../img/acmblack.png}
        \hspace{15pt}
        \includegraphics[width=0.3\textwidth]{../img/qrcode.png}
      \end{center}

    \end{minipage}
  };
  \node[] at (1.6,-1.4) {Únete a ACM};
\end{tikzpicture}


\begin{tikzpicture}
\node [mybox] (box){%
\begin{minipage}{0.3\textwidth}

\begin{verbatim}
# Un pipe `|` redirige la salida de un proceso
#  a la entrada de otro
proc1 | proc2 | proc3

./h.py > o.txt     # stdout a o.txt
./h.py >> o.txt    # stdout a o.txt, añadir
./h.py 2> err.log  # stderr a err.log
./h.py 2>&1        # stderr a stdout
./h.py 2>/dev/null # stderr a null
./h.py &>/dev/null # stdout y stderr a (null)

# Cotenido de foo.txt a stdin de h.py
python h.py < foo.txt
\end{verbatim}

\end{minipage}
};
\node[fancytitle, right=10pt] at (box.north west) {Entrada/Salida estándar};
\end{tikzpicture}


\begin{tikzpicture}
\node [mybox] (box){%
    \begin{minipage}{0.3\textwidth}

      \begin{tikzpicture}
\node [mybox] (box){%
  \begin{minipage}{0.9\textwidth}

\begin{verbatim}
less a.txt   # Vim-like, solo lectura
  /pattern   # Buscar patrón
    n        # Siguiente ocurrencia
    N        # Anterior ocurrencia
  v          # Abrir en el editor
\end{verbatim}

    \end{minipage}
};
%------------ NAME ---------------------
\node[fancytitle, right=10pt, inner sep=5pt] at (box.north west) { less };
\end{tikzpicture}

\begin{verbatim}
cat      # Muestra el contenido
head     # Muestra las primeras líneas
tail     # Muestra las últimas líneas
tail -f  # Va actualizando la salida

touch f1      # Crea el fichero vacío f1

grep hola file.txt    # Líneas con "hola"

#Equivalente a lo anterior
cat f1 | grep hola

wc f1        # Número de palabras
cat f1 | wc  # Número de palabras

# Líneas con "int"
cat *.java | grep int | wc -l

# Muestra el contexto en N líneas
cat *.java | grep -C 3
\end{verbatim}

    \end{minipage}
};
%------------ NAME ---------------------
\node[fancytitle, right=10pt] at (box.north west) { Lectura de ficheros };
\end{tikzpicture}

\begin{tikzpicture}
\node [mybox] (box){%
    \begin{minipage}{0.3\textwidth}

\begin{verbatim}

NAME="ACM"

echo NAME       # NAME
echo $NAME      # ACM
echo "$NAME"    # ACM
echo '$NAME'    # $ACM
echo "${NAME}"  # ACM

if [ -z "$string" ]; then
  echo "String vacía"
elif [ -n "$string" ]; then
  echo "string no está vacía"
fi

git commit && git push
git commit || echo "Commit failed"

{A,B}           # A B
{A,B}.java      # A.tex B.java
*.java          # Todos lo que acabe en .java

for i in /etc/rc.*; do
  echo $i
done

$#   # Número de argumentos
$*   # Todos los argumentos
$@   # Todos los argumentos desde el primero
$1   # Primer argumento

# https://devhints.io/bash
\end{verbatim}

    \end{minipage}
};
%------------ NAME ---------------------
\node[fancytitle, right=10pt] at (box.north west) {BASH};
\end{tikzpicture}

\begin{tikzpicture}
\node [mybox] (box){%
    \begin{minipage}{0.3\textwidth}
\begin{verbatim}
# Empaqueta los ficheros
tar cf target.tar f1 f2

# Empaqueta y comprime con gzip
tar czf target.tar.gz f1 f2

# Extrae en la carpeta
tar xf source.tar -C folder

# Extrae un archivo comprimido en .
tar xzf source.tar.gz

unzip source.zip    # Extrae un .zip
uzip -l source.zip  # Lista el contenido
\end{verbatim}

    \end{minipage}
};
%------------ NAME ---------------------
\node[fancytitle, right=10pt] at (box.north west) { Archivos comprimidos };
\end{tikzpicture}


\begin{tikzpicture}
\node [mybox] (box){%
    \begin{minipage}{0.3\textwidth}

\begin{verbatim}
xrandr      # Muestra las pantallas

# Salida DP1 en el modo 1920x1080
xrandr --output DP1 --mode 1920x1080

# Consulta por un modo para la resolución dada
cvt 1920 1080

# Crea un modo (usando la salida de cvt)
xrandr --newmode "1920x1080_60.00" [...]

# Añade el modo a una pantalla
xrandr --addmode DP1 "1920x1080_60.00"

# Intenta aplicar la mejor resolución posible
#  a las pantallas que detecte
xrandr --auto
\end{verbatim}

    \end{minipage}
};
%------------ NAME ---------------------
\node[fancytitle, right=10pt] at (box.north west) { Pantallas };
\end{tikzpicture}

\begin{tikzpicture}
\node [mybox] (box){%
    \begin{minipage}{0.3\textwidth}
\begin{verbatim}
ssh user@host    # Login con user en host
ssh host         # Login con $(whoami) en host
ssh -p 2222 host # Usar el puerto 2222
\end{verbatim}

\begin{tikzpicture}
\node [mybox] (box){%
  \begin{minipage}{0.9\textwidth}

\begin{verbatim}

Host triqui
HostName triqui2.fi.upm.es
         User   wXXXXXX
         Port   22
\end{verbatim}
  \end{minipage}
};
%------------ NAME ---------------------
\node[fancytitle, right=10pt, inner sep=5pt] at (box.north west) { .ssh/config };
\end{tikzpicture}

\begin{verbatim}
ssh triqui       # Usa la configuración

ssh-keygen -t rsa -b 4096 -C "comment/email"

ssh-copy-id -i ~/.ssh/id_rsa host
\end{verbatim}

\end{minipage}
};
%------------ NAME ---------------------
\node[fancytitle, right=10pt] at (box.north west) { SSH };
\end{tikzpicture}

\begin{tikzpicture}
\node [mybox] (box){%
    \begin{minipage}{0.3\textwidth}

\begin{verbatim}
bash
zsh

# Personalización: oh-my-zsh
# https://github.com/robbyrussell/oh-my-zsh
# Configuración en .bashrc .zshrc
\end{verbatim}

\begin{tikzpicture}
\node [mybox] (box){%
  \begin{minipage}{0.9\textwidth}

\begin{verbatim}
alias vim='emacsclient -nw -c -a ""'
\end{verbatim}
  \end{minipage}
};
%------------ NAME ---------------------
\node[fancytitle, right=10pt, inner sep=5pt] at (box.north west) { .zshrc };
\end{tikzpicture}

    \end{minipage}
};
%------------ NAME ---------------------
\node[fancytitle, right=10pt] at (box.north west) { Terminales };
\end{tikzpicture}

\begin{tikzpicture}
\node [mybox] (box){%
  \begin{minipage}{0.3\textwidth}

\begin{verbatim}
















































\end{verbatim}

    \end{minipage}
};
%------------ NAME ---------------------
\node[fancytitle, right=10pt] at (box.north west) { Notas };
\end{tikzpicture}


\end{multicols*}
\end{document}